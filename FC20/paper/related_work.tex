\paragraph{Limitations of Existing Solutions.}
To the best of our knowledge, there is no related work on the problem of the decentralization of software updates in the context of blockchain systems in a holistic manner, i.e., taking into consideration all phases in the lifecycle of a software update. 
Bitcoin \cite{bitcoin}, Bitcoin Cash \nnote{citation?}, Ethereum \cite{ethereum} and Zcash \nnote{citation?} use a \say{social governance} scheme, in which decisions on update proposals is reached through discussions on social media. This type of informal guidance is  too unstable and prone to chain splits, or prone to becoming too de-facto centralized \cite{buterin_on_gov}. There exist blockchain systems \cite{dash}, \cite{decred} that adopt a decentralized governance scheme, in which the priorities, as well as the funding of update proposals is voted on-chain as part of a maintenance protocol. However, these proposals do not follow a holistic approach to the decentralization software updates problem. Instead, their focus is merely on the ideation phase in the lifecycle of a software update, where an update proposal is born as an idea, and the community is called to accept, if it will be funded, or rejected. These solutions do not deal with the residual phases in this lifecycle, which pertain to the approval of the source code correctness and the authenticity of the binaries that will be distributed for downloading, the maintenance of the code-base and the activation of the changes. That is why in the above cases, there exist a central authority (or group) that assumes the role of the source code maintainer. A similar approach is proposed by Bingsheng et al. \cite{cryptoeprint:2018:435}, where a complete treasury system is proposed for blockchain systems, in which liquid democracy / delegative voting is followed. We also follow the approach of voting delegation, when technical expertise is required, in order to reach a decision for an update proposal. Similarly, Bingsheng's work is focused on the treasury system, which covers only the initial phase in the lifecycle of an update proposal.


%Good examples of governance:
%\begin{itemize}
%\item Treasury paper (https://eprint.iacr.org/2018/435)
%\item Dash Governance  - https://www.dash.org/governance/ 
%\item Decred Governance - https://docs.decred.org/governance/consensus-rules-voting/
%\end{itemize}
%
%Bitcoin and Ethereum use a \enquote{Social Governance} scheme, which basically means that decisions on update proposals is reached through discussions on social media. This type of informal guidance is  too unstable and prone to chain splits, or prone to becoming too de-facto centralized [https://vitalik.ca/general/2017/12/17/voting.html]
