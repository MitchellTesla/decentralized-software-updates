\section{Update Logic} \label{updlogic}
%\nnote{
%Activation Phase
%the activation phase is not a re-approval phase, it is just there to guard against chain split.s
%It is not relevant to UPs that dont impact consensus (adoption threshold = 0%)
%For those that do impact, adoption threshold = the honest stake threshold assumption of the consensus protocol. We make the assumption that if a UP is approved then all honest stake will eventually upgrade.
%To mitigate risk of chain split by accident (activation too early), we propose the concept of the activation lag.
%}
%\paragraph{What is Update Logic}
%Update Logic = Metadata-Driven Software Updates
One size does not fit all and surely, all software updates are not the same. There are software updates with totally different characteristics that require a totally different update logic to be applied. For example such characteristics could be:
%We have pointed out many times that software updates are not all the same. There are many different perspectives for viewing SUs, which call for a specialized \emph{software update policy}. Software updates can be distinguished by the following dimensions:
\begin{itemize}
\item The reason of change (a bug-fix, or a security-fix versus a change request, or a new feature request).
\item The priority of change (e.g., high/medium/low).
\item Impact of change (e.g., whether it impacts the consensus protocol, or not).
\item Type of protocol change (hard/soft/velvet\footnote{See Zamyatin et. al. \cite{velvet}} fork)
\item The complexity of deployment (e.g., high/medium/low)
%\item Size of change (e.g., man-effort required to be implemented).
%\item Platform-specific change (Linux, Windows MacOS etc.)
\end{itemize}
These and many other characteristics comprise the essential context of a software update and should be clearly described in the respective metadata, which are submitted along the SIPs, or UPs and then reviewed and approved by the stakeholders. We distinguish several categories of metadata that result to a holistic view of a software update. The proposed list of update metadata can be found in the appendix \ref{appxmetadata}. 

We propose a software update logic that is \emph{metadata-driven}. A logic that distinguishes one SU from another and applies the appropriate \emph{update policy} and evaluates the required \emph{update constraints} (cf. appendix \ref{appdxupdcons}) based on these metadata. 
%We have seen from the previous sections that the software update metadata are submitted along with the SIP or UP and are approved for their correctness by the stakeholders. Thus we can assume that these metadata safely set the software update \emph{context}, in which our update logic will be applied. 
In this section, we describe our proposal for achieving such a \emph{metadata-driven}, decentralized software update mechanism by means of \emph{update policies}.

%\subsection{Decentralized Software Update Metadata} 
%A software update is inherently accompanied by meta-information that describes the update and sets it into the appropriate context. Therefore, we could say that every software update comprises a rich set of update meta-data that ultimately should drive the whole upgrade process. We distinguish several categories of metadata that result to a holistic view of a software update. The proposed list of update metadata can be found in the appendix \ref{appxmetadata}. 

% The list of metadata categories presented next is indicative and aims at justifying the concept, therefore it is by no means complete, or restrictive in any way:
%We have described different aspects of meta-information that could accompany a software update. The purpose was not to propose a complete list of metadata but rather to point out how important is the software update mechanism to be metadata driven. In the following subsections, we provide examples of such exploitation of software update metadata in the phase of activation.

\subsection{Update Policies} 

An \emph{update policy} is a method to apply a customized activation of a software update driven by its metadata (i.e., by the software update context). For example, it is reasonable to assume that a high priority security fix must be activated with a different speed than a \say{nice to have} new feature. At the same time, 
%for software updates that impact the consensus protocol, we need to be cautious not to cause a chain split by a premature activation of a software update. 
 software updates with a complex deployment process should take a longer time to activate. Therefore, there exist software updates for which we need to go fast and others for which we need to be cautious and slow down.

We have discussed delegation for special categories (cf. appendix \ref{appxdelspecial}) as a method to speed-up (indirectly) the time to activation for a software update by delegating the approval of such software updates to a specialized group of experts. Moreover, in the previous section we have discussed the concepts of the adoption threshold and the safety lag as the main condition to activate the changes of a software update.

%Here we want to introduce also the concept of the \emph{safety lag}. %In the following subsections, we discuss two other factors that impact directly the activation speed of a software update and thus enable update policies and also propose ways to guard against chain splits.

%\subsubsection{The Safety Lag}
%To mitigate the risk of a too-early activation, %either \say{by-accident} or by-attack, 
%we propose the concept of the \emph{safety lag}. The safety lag is a metadata-driven artificial delay, which is imposed on the activation of a change, in order to give time to the honest stake to upgrade and thus protect against a too-early activation. In other words, even though  the adoption threshold of signals might be met, the activation of the changes does not take place, it is stalled, until the safety lag period passes.

%The safety lag is determined by two things: a) the deployment complexity of a software update and b) the type of the consensus protocol change (soft/hard fork\footnote{Velvet forks \cite{velvet} do not impact the consensus by causing forks and thus can activate immediately after an upgrade.}).
%For example, there might be software updates that entail a very complex deployment process; one that even maybe a hardware upgrade is required before the software upgrade. In such a case, the safety lag must ensure plenty of time to the stakeholders to upgrade. Similarly, a hard fork type of change, should trigger a greater safety lag than that of a soft fork type of change. 
%We propose both of these to be recorded as important characteristics of a decentralized software update's metadata (cf. appendix \ref{appxmetadata}). This information will drive the choice of the length of the safety lag and enable a metadata-driven update policy. %Of course, these metadata have been approved for their correctness by the stake majority in the Ideation phase.

%\subsubsection{Examples of Update Policies}
 In table \ref{update-policies2-table}
 %\ref{update-policies-table}
, we present an indicative example of an update policy that is driven by the software update context. In this example, an update policy is expressed as the triple: (Delegation\footnote{\say{D} stands for (standard) delegation while \say{Sp.D} stands for special category delegation. }, Adoption Threshold $\tau_A$, Safety Lag $T_s$). We see for different types of deployment complexities (High/Medium/Low) and approved and agreed priorities (High/Medium/Low), the proposed update policy that will achieve our activation goal. We see the possible values of both the adoption threshold and the safety lag to be expressed in three distinct bands (High/Medium/Low). Of course, all the relevant constraints for these values that we have discussed in the previous section, in order to ensure the security of the activation, should hold.

For example, a low priority software update with a high deployment complexity, should be activated with a maximum delay (top left cell). At the opposite end (bottom right cell), a high priority software update (e.g., a security fix) that also 
bears a low deployment overhead, should be activated with a minimum of delay. 

%We see, for different examples of SUs, how to achieve the specified activation goal (speed-up, or slow-down) with a specific update policy. Note that the various characteristics of the software update, such as the type of the change, the priority/criticality, etc., that appear in the first column and drive the deployment goal, have been described in the SU metadata and have been reviewed and approved by the stakeholders during the Ideation and the Approval phases.
 
% In our proposal, we enable update policies and at the same time guard against chain splits in three ways: a) With the delegation to expert pools, b) with appropriate adjustment of the \emph{adoption threshold} and c) with the use of the \emph{activation lag}.
% We have discussed in the delegation section why it is important critical categories of software updates to be delegated to expert pools and why this will enable overall a faster time to activation. In the next subsections, we discuss the adoption threshold and the activation lag.

%An update policy is a way to customize the activation speed of a SU based on the type of the SU, which is deduced by the SU's metadata. We want to follow a metadata-driven activation approach. 
%An update policy can be enabled by two things:
%A) Delegation to expert pools
%B) adoption threshold (activation phase) and activation lag (activation phase)
%Activation lag is determined by
%Deployment complexity
%Soft/Hard fork type of change

%\begin{table}[h!]
%\centering
%%\begin{tabular}{|l|l|l|l|l|}
%\begin{tabu} to 1.0\textwidth {||X[3.5l] | X[l] | X[2l] | X[0.5l] | X[l] ||}
%\hline
%\textbf{Type of SU} & \textbf{Goal} & \textbf{Delegation} & $\tau_A$ & \textbf{Safety Lag} \\
%\hline
%Critical security-fix with low deployment complexity & Speed-up & Special category delegation & $x$ & 0 \\
%\hline
%Critical security-fix with high deployment complexity & Speed-up & Special category delegation & $x$ & High \\
%\hline
%Hard fork type of consensus impact & Delay & Delegation for specific SU  & $h_a$ & High \\
%\hline
%Soft fork type of consensus impact & Small delay & Delegation for specific SU  & $x$ & Medium \\
%\hline
%High priority, no-consensus impact change request with high deployment complexity & Speed-up & Default Delegation & $x$ & 0 \\
%\hline
%Low priority, no-consensus impact change request with high deployment complexity & Delay & Default Delegation & $h_a$ & High \\
%\hline
%%\end{tabular}
%\end{tabu}
%\caption{Examples of different update policies}
%\label{update-policies-table}
%\end{table}

%\begin{table}[h!]
%\centering
%%\begin{tabular}{|l|l|l|l|}
%\begin{tabu} to 1.0\textwidth {||X[0.6l] | X[l] | X[l] | X[l] ||}
%\hline
%\textbf{Hard Fork} & High Delay\newline(D, $h_a$, High) & Medium Delay\newline(D, $h_a$, Medium) & Low Delay\newline(D, $h_a$, Low) \\
%\hline
%\textbf{Soft Fork} & Medium Delay\newline(D, $x$, High) & Low Delay\newline(D, $x$, Medium) & Low Delay\newline(D, $x$, Low) \\
%\hline
%\textbf{No Fork} & Low Delay\newline(D, $0$, $0$) & No Delay\newline(Sp.D, $0$, $0$) & No Delay\newline(Sp.D, $0$, $0$) \\
%\hline
%\textbf{Priority} & \textbf{LOW}  & \textbf{MEDIUM} & \textbf{HIGH} \\
%\hline
%%\end{tabular}
%\end{tabu}
%\caption{Examples of different update policies based on the SU context.}
%\label{update-policies2-table}
%\end{table}

\begin{table}[h!]
\centering
%\begin{tabular}{|l|l|l|l|}
\begin{tabu} to 1.0\textwidth {||X[0.7l] | X[c] | X[1.01c] | X[1.01c] ||}
\hline
\textbf{HIGH} & \textbf{High Delay}\newline(D, High, High) & (D, Medium, High) & (Sp.D, Low, High) \\
\hline
\textbf{MEDIUM} & (D, High, Medium) & (D, Medium, Medium) & (Sp.D, Low, Medium) \\
\hline
\textbf{LOW} & (D, High, Low) & (D, Medium, Low) & \textbf{Low Delay}\newline(Sp.D, Low, Low) \\
\hline
\textbf{Complexity/ Priority} & \multicolumn{1}{l|}{\textbf{LOW}} & \multicolumn{1}{l|}{\textbf{MEDIUM}} & \multicolumn{1}{l||}{\textbf{HIGH}} \\
\hline
%\end{tabular}
\end{tabu}
\caption{An example of an update policy based on the SU context.}
\label{update-policies2-table}
\end{table}

