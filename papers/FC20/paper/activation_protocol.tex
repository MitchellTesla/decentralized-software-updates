\subsection{Candidate Protocols for PoS Ledgers}
We recall that the value $j$ in the command $(\activate,\cdot, j)$ refers to the position of the latest block of $\ledger_1$ 

Let $\ledger_1$ be the ledger that the parties are currently running and $\ledger_2$ be the new ledger.
In the description of our update protocol we assume that there is a time $t^\upd$ such that $\asmp_2[t]$ holds for all $t\geq t^\upd$.\footnote{We could require $\asmp_2$ to
hold only for bounded interval of time and our protocol would still work if this interval is large enough.} To ensure that $\asmp_2$ actually holds we could use
the approach proposed in Sec.~\ref{se:u}. 
We denote with $\activeph$ the honest parties that received the command $(\activate,\cdot, j)$ Upon receiving $(\activate,\cdot, j)$, each party in $\activeph$ does the following steps.
\begin{enumerate}
	\item Wait until the block $(j+k)$-th block is added to $\ledger_1$. Let $B_{j+k}$ be this block.
	\item Use $B_{j+k}$ as the genesis block for $\ledger_2$.
\end{enumerate}







In the 
We describe our protocol under the assumption the number of honest parties that receive the input $(\activate,\cdot)$ is such that $\asmp_2$ holds.

 that all the honest parties will eventually activate if a proposal has been accepted. Moreover, we assume that the proposal is with
respect to a ledger $\ledger_2$ that is secure (i.e., if $\asmp_2$ holds then liveness and consistency hold). We also assume that $\asmp_1=\asmp_2$. Let $t_\accept$ be the time at which the proposal 
for $\ledger_2$ is accepted.
In this protocol when a party $P_i$ receives the command $(\activate, \ledger_2)$ at some time $t_i$ then it creates a transaction $\txa_i$ on $\ledger_1$ that contains the hash of the last block of $P^\ledger[t]$. Let $B_i$ be this block. $\tx_i$ is used by $P_i$ to signal the fact that he is ready to run $\ledger_2$ using $B_i$ or a block
with greater height of $B_i$ as the genesis block if a sufficient number of activation transactions appear on $\ledger_1$. 
More precisely, $P_i$ starts running $\ledger_2$ if the sum of the stake associated to the
verification keys of all the activation transactions is above a threshold $\tau$. Whenever this threshold is reached all the parties that activated (the parties in $\activep$)
start running $\ledger_2$ using $B_j$ as the genesis block where $B_j$ has the most height among all the blocks of $\ledger_1$ hashed in the activation transactions $\txa_i$
for all $P_i\in\activep$.
We prove that this protocol is secure for any $\tau$ 
\mnote{$\Delta_1$ and $\Delta_2$ should be function of $\tau$.}

Let us consider the simple case where $\tau=1.0$. In this case we can claim that $\asmp_2$ holds (assuming that $\asmp_1$ holds).
Moreover, the parties $\activep$ has an agreement on what is the genesis block for $\ledger_2$. We recall that we assume that any block of $\ledger_1$ can be used as
a genesis block for $\ledger_2$ under the condition that $\ledger_1$ is secure (see Def.~\ref{def:genesis} for a formal definition connecting genesis block).
Under this condition the security of $\ledger_2$ \mnote{should} follows. 

 
This argument can be extended to any threshold $\tau$. Indeed, since we are assuming that the honest nodes will activate
Suppose that this threshold is any value that that makes $\asmp_2$ valid.
