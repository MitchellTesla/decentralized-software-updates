\subsection{Candidate Protocol}

Let $\ledger_1$ be the ledger that the parties are running and $\ledger_2$ be a new ledger.
In our construction we assume that $\ledger_1$ and $\ledger_2$ have same common-prefix parameter $k$, same chain-growth parameter $(\cg, s)$ and that they are secure under the same assumption $\asmp_1=\asmp_2$. We also assume that a block $B$ of $\ledger_1$ can be transformed into a \emph{special genesis block} $B'$ for $\ledger_2$. $B'$ is special because not only
it represents a valid genesis block for $\ledger_2$, but it also points to the state of $\ledger_1$ up to the block $B$. 
We use this a special genesis block to guarantee that 1) the state of $\ledger_1$ at the end of the activation is moved into the state of the new ledger
and 2) that we can use a block of $\ledger_1$ to bootstrap $\ledger_2$.

At a very high level our protocol works as follows. When enough parties have received the command $(\activate,\cdot)$ such that $\asmp_2$ holds,
a block of $\ledger_1$ is chosen as the special genesis block for $\ledger_2$. At this point the parties start running $\ledger_2$ and gradually abandon 
$\ledger_1$.

In the description of our update protocol we assume that there is a time $t^\upd$ such that $\asmp_2[t]$ holds for all $t\geq t^\upd=t_{P_1}+\Delta_1$.\footnote{We could require $\asmp_2$ to
hold only for bounded interval of time and our protocol would still work if this interval is large enough.}
We also assume that the parties in $\activep$ knows an index $j$ such that when the $j$-th block is added to the state of $\ledger_1$ then $\asmp_2$ holds. 
More formally, at time $t^\upd$ the $j$-th block is part of $\ledger_1^{P_i}$ for all $P_i\in\activep$.
To ensure that $\asmp_2$ actually holds and to keep all the honest parties synchronized on the time $t^\upd$ (and on the index $j$) we use the approach proposed in Sec.~\ref{se:informal}.
In more details $t^\upd$ represents either the time at which the threshold is met or it represents the time in which the safety lag period ends.
 

%\nnote{We can arrive at moment $t^\upd$, where $\ledger_2$ becomes secure, in two different ways: a) either the adoption threshold is met or b) the $j$-th block is added to the state of $\ledger_1$. If a) happens, then we dont need to wait for the $j$-th block to arrive. This last sentence does not "integrate" well with section 3.1}

We denote our update protocol with $\Pi$. In this, each honest party $P_i\in\activep$ at time $t^\upd$ executes the following steps.

\begin{enumerate}
	%\item Upon receiving the input $(\activate, \cdot)$ wait until the $j$-th block of $\ledger_1$ becomes part of all the honest parties' ledger state (i.e., wait until $B_{j}$ becomes part of $\ledger_1^{P_i}[t]$ 
	%for all honest $P\in\activep$) and let $t^\upd$ be this time. \nnote{But if the adoption threshold is met, then we dont need to wait until $t^\upd$}
	\item Run $\ledger_1$ and when the block $(j+k)$-th block $B^i_{j+k}$ becomes part of $\check \state^{P_i}[t]$ for some $t\geq t^\upd$
	start running also $\ledger_2$ using ${B^i}'$ as the genesis block (where ${B^i}'$ is generated using $B^i_{j+k}$).\footnote{Note that the block
	 $B^i_{j+k}$ that $P_i$ sees might not become part of the $\state_1$ state.}
	\item Run $\ledger_2$ (and keep running $\ledger_1$) until $B'$ becomes part of $\state_2^{P_i}[t]$ 
	for each honest $P\in\activep$ for some $t\geq t^\upd$.
	\item Stop running $\ledger_1$.
\end{enumerate}



\begin{theorem}
Let $\ledger_1$ and $\ledger_2$ be two ledgers that are secure under the assumption $\asmp_1=\asmp_2$ and
that have common-prefix parameter $k$ and chain-growth parameter $(\cg, s)$ then
 $\Pi$ is a $(\Delta_1,\Delta_2)$-secure update system with $\Delta_1=k\cg^{-1}$ and $\Delta_2=k \cg^{-1} + k \cg^{-1}$.
\end{theorem}

To simplify our proof we introduce the notion of \emph{canonical scenario} for the the ledger $\ledger_2$. In a canonical scenario the ledger
$\ledger_2$ is executed in the standard way. More precisely, 
we assume the existence of a genesis block and that $\asmp_2[t]$=1 for all $t\geq 0$. Let $\parties$ be the set of parties that is running $\ledger_2$. Also, 
let $t_j$ be the smallest time slot in which $B_j$ appears in $\state_2^{P_i}[t]$ for each $P_i\in\parties$ and let $t_{j+k}$ be smallest time slot in which $B_{j+k}$ appears in 
$\state_2^{P_i}[t']$ for each $P_i\in\parties$.
We are now ready to prove the security of $\Pi$.

\begin{proof}

By assumption we have that $\asmp_2[t]=1$ for all $t\geq t^\upd$. From the description of $\Pi$ we can claim that $t^\upd \leq t_{P_1}+k\cg^{-1}=t_{P_1}+\Delta_1$.
That is, when an honest party is activated she waits to be sure that also the other honest parties are activated. Hence, from the chain-growth and the common-prefix parameters
each party needs to wait at most $\Delta_1=k\cg^{-1}$ time slots.

From the moment when $\asmp_2$ becomes true the activation process takes $\Delta_2=k \cg^{-1} + k \cg^{-1}$  time slots more to be completed.
This is because the parties need to wait that $k$ blocks are generated in $\state_1$ and that $k$ blocks are generated in $\state_2$. 
That is, the parties need to wait the $(j+k)$-th block to appear in $\state_1^P[t]$ for each honest $P\in\activep$) and the \emph{special genesis block} of $\ledger_2$ (that is generated using the block $(j+k)$-th block of $\ledger_1$) to appear 
in $\state_2^P[t]$ for each honest $P\in\activep$. Given that a block in $\ledger_1$ ($\ledger_2$) takes at most $\cg^{-1}$ time slots then we have that $\Delta_2=k \cg^{-1} + k \cg^{-1}$.


In the moment that a \emph{candidate} block $B^i_{j+k}$ becomes available to an honest party $P_i\in\activep$ (i.e., $B^i_{j+k}$ is part of $\check \state_1^{P_i}$) then she starts running $\ledger_2$
using ${B^i}'$ which is computed from $B^i_{j+k}$ as described earlier (we recall that at this time slot the assumption $\asmp_2$ holds).
Let $t'$ be smallest time slot in which $B_{j+k}$ appears in $\state_2^{P_i}[t']$ for each $P_i\in\parties$.
If we take the execution of the protocol from time $t^\upd$ and $t'$ this can be seen as a canonical execution of $\ledger_2$ given
that the parameters of $\ledger_1$ and $\ledger_2$ are the same. The only difference between this and the canonical scenario is
that the blocks $B_{j}, B_{j+1},\dots, B_{j+k}$ are generated using $\ledger_1$, but this does not represents an issue since we are assuming that
any block of $\state_1$ can be turned into a block of $\state_2$. 


\ignore{
We now observe that at time $t^\upd$ we are in the canonical scenario since the assumption $\asmp_2$ holds and the honest parties
have no information of the candidate block for the position $j+k+1$. The first with the canonical scenario and this case is that the parties will start running 
$\ledger_2$ (which is secure) on the $(j+k+1)$-th.
The other difference is that there is not a genesis block for $\ledger_2$. Indeed the block $j+k$ it might no be stable. To solve this problem we keep running $\ledger_1$
until the block 
}
%We also note that when $\asmp_2$ becomes true the adversary, and potentially all the honest parties in $\activep$, not know what the candidate blocks for the position $j+k$.
%This because we are guaranteed that at least one of the blocks will extend $j$ is honest and therefore not known to the adversary. The situation in which the 
%parties $\activep$ start running $\ledger_2$ using as potentially different blocks ${B^i}'$ is equivalent to the situation 


\end{proof}
