\section{Conclusion}
\mnote{I suggest to remove this section}
In this paper, we have taken a holistic approach on software updates and have proposed decentralized alternatives for all the phases in their lifecycle - from the ideation phase till the actual activation of the changes on the blockchain system. To this end, we have formally defined what is a secure software update and have proposed a decentralized secure software update protocol that realizes our solution. Moreover, we have departed from the one-size-fits-all approach and have shown, how can we achieve a metadata-driven software update mechanism based on the software update context. Our results can be applied to any stake-based blockchain system and contribute significantly towards the decentralization of their maintenance and governance processes. The biggest open issue that our work did not cover is the definition of an appropriate software updates-specific incentives scheme that would guarantee a smooth operation of our protocol in the presence of stakepools \cite{stakepools}. This, along with the incorporation of our protocol in the Cardano blockchain system \cite{cardano} comprise the main parts of our future work.