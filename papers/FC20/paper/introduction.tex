\section{Introduction}
Software updates are everywhere. The most vital aspect for the sustainability of any software system is its ability to effectively and swiftly adapt to changes; one basic form of which are software updates. Therefore the adoption of software updates is in the heart of the lifecycle of any system and blockchain systems are no exception. Software updates might be triggered by a plethora of different reasons: change requests, bug-fixes, security holes, new-feature requests, various optimizations, code refactoring etc.

%Typically, the main driver for a change will be a change request, or a new feature request by the user community. These type of changes will be planned for inclusion in some future release and then implemented and properly tested prior to deployment into production. In addition, another major source of changes is the correction of bugs (i.e., errors) and the application of security fixes. These usually produce the so-called \say{hot-fixes}, which are handled in a totally different manner (i.e., with a different process that results into different deployment speed) than change requests. In addition, depending on the severity level of the problem, or the priority (i.e., importance) of the change request there are different levels of speed and methods for the deployment of the software update into production. 

More specifically, for blockchain systems, a typical source of change are the enhancements at the consensus protocol level. There might be changes to the values of specific parameters (e.g., the maximum block size, or the maximum transaction size etc.), changes to the validation rules at any level (transaction, block, or blockchain), or even changes at the consensus protocol itself. Usually, the reason for such changes is the reinforcement of the protocol against a broader scope of adversary attacks, or the optimization of some aspect of the system like the transaction throughput, or the storage cost etc.

%Finally, there are also more radical changes, which are usually caused by the introduction of new research ideas  and the advent of new technology, which becomes relevant. These type of changes usually introduce new concepts and are not just enhancements and thus trigger a major change to the system.

\paragraph{Context of this paper.} 
In this paper, our focus is on the software update mechanism of stake-based blockchain systems. We depart from the traditional centralized approach of handling software updates, which is the norm today for many systems (even for the ones that are natively decentralized, like permission-less blockchain systems) and try to tackle common software update challenges in a decentralized setting. We consider the full lifecycle of a software update, from conception to activation and propose \say{decentralized alternatives} to all phases. Essentially, we introduce a decentralized \emph{maintenance} approach for stake-based blockchain systems.

%\paragraph{Context of this paper.} In this paper, our focus is on the update mechanism of permissionless stake-based blockchain systems. We try to overcome known shortcomings of today's update methods for blockchain systems and propose a logical architecture for an update mechanism for stake-based ledgers. Our architecture covers all the components of a typical blockchain system and demonstrates how can such a mechanism be smoothly incorporated into each layer of the blockchain system. We describe in detail the various enhancements and changes in the existing components and then proceed into describing new components such as the governernance component, or the update-logic component. Our aim is to cover all aspects of software updating, from update proposal submission to update deployment and activation. Finally, we implement our logical architecture into a prototype update system for the Cardano blockchain. We then discuss various implementation issues and the prototype architecture.


\paragraph{Problem Definition.} 
Traditionally, software updates for blockchain systems have been handled in an ad-hoc, centralized manner: somebody, often a trusted authority, or the original author of the software, provides a new version of the software, and users
download and install it from that authority's website. Even if the system follows an open source software development model, and therefore an update can potentially be implemented by anyone, the final decision of accepting, or rejecting, a new piece of code is always taken by the main maintainer(s) of the system, who essentially constitutes a central authority. Even in the case, where the community has initially reached at a consensus for an update proposal (in the form of an \emph{Improvement Proposal} document), through the discussion that takes place in various discussion-forums, still it remains an informal, \say{social} consensus, which is not recorded as an immutable historical event in the blockchain and the final decision is always up to the code maintainer to accept it, or not. Moreover, the authenticity and safety of the downloaded software is usually verified by the digital signature of a trusted authority, such as the original author of the software.

On the other hand, in a decentralized setting software updates should be handled quite differently: An update proposal should be possible to be submitted by anyone, who can potentially submit a transaction to the blockchain. The decision of which update proposal will be applied and which will not, should be taken collectively by the community and not centrally. Thus, the roadmap of the system should be decided jointly. Moreover, this process should not be an informal discussion process, but it should be part of an update protocol and all relevant events generated, to be stored within the blockchain itself, and thus be recorded in the immutable update history of the system. Moreover, in the decentralized setting, the role of the code maintainer, who takes decisions on the correctness of the submitted new code, the versions of the software and also guarantees the validity of the downloaded software, should be replaced by the stake majority. 

Therefore, in the context of software updates for public stake-based blockchain systems, we define the capability to take stake majority-based decisions \mnote{I am not sure
that we should talk about stake majority-based decision. Indeed, we later discuss the possibility of setting the threshold (that defines whether a proposal should be accepted or not) to be anything greater than $52\%$}on: a) the priorities of software updates, b) the correctness of the new code, c) the maintenance of the code-base and d) the authenticity and safety of the downloaded software
%, as part of the consensus protocol and recorded on-chain, while at the same time fulfilling software dependencies requirements, resolving version conflicts, enabling different update policies based on update metadata and avoiding chain splits when updates are activated
, as the \emph{decentralized software updates problem} and this is the problem that we are trying to solve in this paper. In addition, we investigate how we can at the same time fulfill software dependency requirements, resolve conflicts, enable different update policies based on update metadata and avoid chain splits when updates are activated.
\mnote{The latest might become on of the main focus of the paper. We should highlight that.}

%\paragraph{Problem Definition.}  Traditionally, software updates have been handled in an ad-hoc, centralized manner: Somebody, often
%a trusted authority, or the original author of the software, provides a new version of the software, and users
%download and install it from that authority's website. However, this approach is clearly not decentralised, and
%hence jeopardizes the decentralized nature of the whole system: In a decentralized software update mechanism,
%proposed updates can be submitted by anyone (just like anyone can potentially create a transaction in a blockchain).
%The decision of which update proposal will be applied and which wont, is taken collectively by the community
%and not centrally. Thus, the road-map of the system is decided jointly. Moreover, there is no central code
%repository, nor there is some main software maintainer, who decides on the code. All versions of the code
%are distributed and only local copies exist, in the same manner that the ledger of transactions is distributed in
%blockchain. The decentralized software update system must reach a consensus, as to what version is the current
%one (main branch) and which code branch will be merged into main. Finally, the decentralized software update
%system guarantees the authenticity and safety of the downloaded software, without the need to have some central authority
%to sign the code, in order to be trusted


%The traditional way of handling software updates is neither decentralized nor secure
%\begin{itemize}
%\item No standard way to propose updates
%\item Not a decentralized and democratic way to reach at a consensus on update priorities. Only \enquote{social consensus} is reached via social media. This is unstable and prone to chain splits.
%\item No essential auditing and verifiability of the agreement
%\item No standard way to record the immutable history of all these events (proposals,agreement, \item activation of updates)
%\item No standard method for security guarantees for the software installed
%\item No standard way to resolve conflicts and respect dependencies 
%\item No standard Update metadata
%\item One-size-fits-all for all \enquote{types} of changes (bugs,CRs)
%\item Hard-forks are the norm
%\end{itemize}

\import{./}{related_work.tex}

\paragraph{Goal of the paper.} Our primary goal in this paper, is to propose a secure software update mechanism that will enable a decentralized approach to the blockchain software updates problem. We want to examine all phases in the lifecycle of a software update and propose decentralized alternatives that are practical and can be adopted in the real world. These alternatives essentially substitute any \emph{central authority} with the \emph{stake majority}. In order to enable this \emph{decentralization} of a software update, we exploit existing primitives that we combine them, in order to form a novel decentralized software updates protocol. From a security perspective, our proposal ensures that:
%...A safe and secure decentralized software update system. %In other words, a system that will ensure that: 
a) any stakeholder will always be able to submit an update proposal to be voted by the stakeholders community, b) an update proposal that is not approved by the stake majority
\mnote{Also here, let's talk about a threshold $t$ greater than $51\%$ instead of stake majority since this is what we do} will never be applied, c) an update proposal that it is approved by the stake majority will be eventually applied, d) an update proposal that the stake majority decides that it has a higher priority than some other proposal, will take higher priority, e) it provides guarantees for the authenticity and safety of the downloaded software and finally f) it protects against chain splits, during activation of the proposed updates.

\paragraph{Outline of the paper.}
%\noindent\textbf{Outline of the paper}