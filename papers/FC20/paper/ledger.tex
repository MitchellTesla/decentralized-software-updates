\subsection{Ledger Consensus}
Ledger consensus (a.k.a. ``Nakamoto consensus'') is the problem where a set of servers
(or nodes) operate continuously accepting inputs
that are called transactions and incorporate them in a public data 
structure called the {\em ledger}. Clients are able to read the ledger
and submit transactions to be
added to it.
The purpose of ledger consensus is to provide a unique
view to the ledger for the clients. 

The properties that a ledger consensus protocol must satisfy are as follows:

\begin{itemize}

\item {\em Consistency:} This property mandates that if a client queries
an honest node's ledger at round $t_1$  
and receives the response $\mc{L}_1$, 
then a client querying an honest node's ledger at round $t_2\geq t_1 $
 receives a response $\mc{L}_2$ that satisfies $\mc{L}_1 \preceq \mc{L}_2$, 
where $\preceq$ denotes the standard prefix operation. 

\item {\em Liveness:} If a transaction $tx$ is given as input to all honest
nodes continuously for a certain number of rounds denoted by $u$, 
and a client queries any honest node's ledger
after these $u$ rounds, then the node responds with a value
$\mc{L}$ that includes $tx$.

\end{itemize}

