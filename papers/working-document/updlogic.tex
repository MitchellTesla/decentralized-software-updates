\section{Update Logic}
%\nnote{
%Activation Phase
%the activation phase is not a re-approval phase, it is just there to guard against chain split.s
%It is not relevant to UPs that dont impact consensus (adoption threshold = 0%)
%For those that do impact, adoption threshold = the honest stake threshold assumption of the consensus protocol. We make the assumption that if a UP is approved then all honest stake will eventually upgrade.
%To mitigate risk of chain split by accident (activation too early), we propose the concept of the activation lag.
%}
%\paragraph{What is Update Logic}
%Update Logic = Metadata-Driven Software Updates
We have pointed out many times that software updates are not all the same. There are many different perspectives for viewing SUs, which call for a specialized \emph{software update policy}. Software updates can be distinguished by the following dimensions:
\begin{itemize}
\item Reason of change (a bug-fix, or a security-fix versus a change request, or a new feature request).
\item Priority of change (e.g., high/medium/low).
\item Size of change (e.g., man-effort required to be implemented).
\item Impact of change (e.g., whether it impacts the consensus protocol, or not).
\item Type of protocol change (hard/soft/velvet fork)
\item Platform-specific change (Linux, Windows MacOS etc.)
\end{itemize}
We propose a \emph{software update logic} that is \emph{metadata driven}. A \say{logic} that can distinguish one SU from another and apply the appropriate \emph{update policy}. In the section, we describe our proposal for achieving such a decentralized metadata-driven, software update mechanism.

\subsection{Software Update Metadata} 
A software update is inherently accompanied by meta-information that describes the update and sets it into the appropriate context. Therefore, we could say that every software update comprises a rich set of update meta-data that ultimately should drive the whole upgrade process. We distinguish several categories of metadata that result to a holistic view of a software update. The list of metadata categories presented next is indicative and aims at justifying the concept, therefore it is by no means complete, or restrictive in any way:
%\begin{itemize}
%\item
\paragraph{Basic information.}The software update metadata should provide basic information about the software update, such as its title and a basic description. Also, the metadata should include the unique id of the SU, which can be the content hash of the SU and also the \emph{category tag} that will enable delegation for specialization, as we have described above. Other basic information include, the author of the SU, version information and a link pointing to the storage area, where the software update is stored, as well as a link to the metadata itself. Finally, since in our proposal, we distinguish software updates from SIPs to UPs, the basic metadata information should include a flag to separate one from the other.
\paragraph{Justification.} This is a very important part of the metadata of a software update, especially if this is a SIP. It is the information, with which the author of the update will try to convince the rest of the stakeholders on the merit of the proposed software update. Missing information in this part, or unclear justification statements, might cause the rejection of a proposal. The reason, the scope and the expected benefits of a software update must be clearly stated in this part.
\paragraph{Urgency.} This is where the priority requirement of a software update must be declared. If it is a bug-fix for example, then it should specify a severity level, in order to declare the urgency for deployment. Similarly, if it is a change request, a priority specification, will help the stakeholders prioritize the demand.
\paragraph{Consensus impact.} There are software updates that impact the consensus protocol and others that do not. For the former, it is important to declare the type of the change. If the validation rules of the protocol become less restrictive, then this is a \emph{hard fork} type of change. In this case, non-upgraded nodes reject the new type of blocks and thus there is a significant risk for a chain split. If the validation rules become more restrictive, then this is called a \emph{soft fork} type of change. In this case, the non-upgraded nodes accept the new type of blocks but the blocks generated buy these nodes are not accepted by the upgraded nodes. So, if the old nodes do not eventually upgrade, then they can not continue to issue new blocks. Finally, if the new validation rules are neither more restrictive, nor less restrictive, then we have a \emph{velvet fork} \cite{velvet}. This type of change does not entail the risk of a chain split. Naturally, the type of change is really important, because it signifies the risk for a \emph{chain split}. For example, a hard fork type of change entails much more risk for a chain split, in the case of nodes failing to upgrade on time, than a soft fork, or a velvet fork (which has no risk at all).
\paragraph{Implementation.} Next comes meta-information on the implementation of the software update. Possible examples could be, if this software update entails code development, or if it is a parameter change. For the latter, a distinction could be made between static and dynamic parameter changes. Essentially, dynamic parameter changes are \say{code-less} software updates - no deployment of new code is required. An explicit list of protocol parameters affected by the software update must be provided. This will be also exploited for the conflict resolution between software updates, mentioned below. Other aspects of the implementation have to do with the size and the complexity of a software update. Typically an estimation of the required man-effort for its implementation could be included.
\paragraph{Deployment.} The deployment of a software update is really critical, especially for those updates that impact the consensus protocol. This is because a lack of synchronization at the deployment phase might result into a chain split \say{by accident}. This part of the metadata comprises instructions on the deployment process, maybe deployment scripts, or declarative statements of the deployment process (if some automated software provisioning tool is used, e.g., ansible playbooks). Moreover, an estimation of the size of the deployment (in terms of man-effort required), would help to calculate parameters, such as the \emph{activation lag} described later on.
\paragraph{Rollback.} Symmetrically to the deployment, the metadata must include information for an un-install process, in the case of a problem. The type of information provided is similar to the deployment category.
\paragraph{Update constraints.} Software updates do not live in vacuum. They are strongly related to one another. In fact, there exist dependencies between software updates that if they are not followed, then the upgrade will fail. We call all the dependencies and the conflicts between software updates \emph{update constraints}. It is very crucial to clearly define the update constraints of a software update in its metadata. This will illustrate how feasible is a specific software update. For example, a dependency of a SU on another SU, which has not been implemented yet, is a clear indication of non-feasibility. Also, if two candidate software updates both change the value of the same protocol parameter, then they are in direct conflict and some sort of resolution should take place.
\paragraph{Update prerequisites.} In this part, we define other prerequisites apart from the ones that have to do with other software updates that we have discussed before and therefore, we can consider them as prerequisites from \say{external} factors. For example, platform requirements, software requirements (e.g., external libraries), or specific hardware requirements, in order for a software update to be applied successfully, should be mentioned in this section of the metadata.
\paragraph{Budget information.} Last but not least, comes the budget information. This is extremely useful information, especially, if the update system is backed up by a treasury system \cite{treasury} that will assume the funding of the implementation of the software update.
%\end{itemize}

We have described different aspects of meta-information that could accompany a software update. The purpose was not to propose a complete list of metadata but rather to point out how important is the software update mechanism to be metadata driven. In the following subsections, we provide examples of such exploitation of software update metadata in the phase of activation.


%\nnote{categories of metadata:
%- basic info (name, description, content hash, author description, version info, url, Category tag)
%- Justification of change (bug-fix, security-fix, CR, benefits from change etc.
%- Urgency of change (priority, level of severity)
%- Consensus impact (hard/soft/velvet/no-impact
%- implementation (method: dynamic/static parameter change, code development, size estimate: man-effort)
%- Change size (man-effort estimated)
%- Deployment (complexity man-effort estimated, instructions, ansible playbook)
%- Uninstall (instructions, ansible playbook)
%- Budget info
%- Update Constraints (dependencies to other SUs, prerequisites, conflicts with other SUs)  
%- Update prerequisites (Platform, HW requirements)}

%We need to define the set of update metadata that we need to maintain, in order to implement various \emph{update policies}, achieve conflict resolution, respect update dependencies and in general support the full scope of the Updating Process. These metadata will be included in the Update Proposal (see the Update Manifest below)

\subsection{Update Policies} 
An \emph{update policy} is a method to apply a customized activation of a software update driven by its metadata (i.e., by the software update context). For example, it is reasonable to assume that a high priority security fix must be activated with a different speed than a \say{nice to have} new feature. At the same time, for software updates that impact the consensus protocol, we need to be cautious not to cause a chain split by a premature activation of a software update.
In the following subsections we discuss the factors that impact the activation speed of a software update and thus enable update policies and propose ways to guard against chain splits.
 
% In our proposal, we enable update policies and at the same time guard against chain splits in three ways: a) With the delegation to expert pools, b) with appropriate adjustment of the \emph{adoption threshold} and c) with the use of the \emph{activation lag}.
% We have discussed in the delegation section why it is important critical categories of software updates to be delegated to expert pools and why this will enable overall a faster time to activation. In the next subsections, we discuss the adoption threshold and the activation lag.

%An update policy is a way to customize the activation speed of a SU based on the type of the SU, which is deduced by the SU's metadata. We want to follow a metadata-driven activation approach. 
%An update policy can be enabled by two things:
%A) Delegation to expert pools
%B) adoption threshold (activation phase) and activation lag (activation phase)
%Activation lag is determined by
%Deployment complexity
%Soft/Hard fork type of change

\subsubsection{The Adoption Threshold}
In the lifecycle section, we have described how a software update, after being voted as a SIP, it is submitted again during the approval phase in the form of a UP (source code bundled with metadata), in order to be approved. We make the assumption that if a UP is approved, then all honest stake will eventually upgrade. However, if we want to be fast and flexible in our updating policies, then we cannot wait until all honest stake upgrades. What is the minimum necessary percent of stake to have upgraded, before the actual activation of the change takes place, in order to avoid a chain split? The \emph{adoption threshold} is used in the activation phase and corresponds exactly to the minimum percent of stake that is necessary to have signaled upgrade readiness, before the actual activation of a software update takes place. It is essentially a synchronization mechanism that ensures that a sufficient percent of stake has upgraded and thus it is safe to actually activate the changes. It is therefore a guard against chain splits. Please note that the adoption threshold is only relevant for software updates that impact the consensus protocol. For all other software updates the activation can take place immediately after the upgrade.

Let us assume that the adoption threshold of our software updates protocol is called $\tau_A$. %In order to enable update policies, we need to be able to adjust $\tau_A$ based on each software update's metadata. 
What would be the appropriate values for $\tau_A$? Take into account that by adjusting the value of $\tau_A$, we risk to cause a chain split for two distinct reasons: a) A too-low value of $\tau_A$ might result to a \emph{too-early activation}, which will result to the partition of the honest stake in two and a potential chain split\footnote{A partition of the honest stake also undermines the security of the underlying consensus protocol which assumes a minimum threshold $x$ of honest stake.}  and b) a too-high value of $\tau_A$ might result to a \emph{too-late activation}, giving the opportunity to adversaries to block an activation by refusing to signal. %Clearly the former is a safety problem, while the latter is a security problem. 
What is the allowable range of values for $\tau_A$, in order to mitigate these risks?

As we have seen from the voting section, we assume that the honest stake threshold of our software updates protocol is $h$. Also, in the same section we have seen that, if $x$ is the honest stake threshold of the underlying consensus protocol then $h \geq x$. Let us also assume that the actual percent of honest stake is $h_a$. Then, since $h$ is the minimum threshold, the following will hold: $ h_a \geq h$. Therefore, $h_a \geq x$. For the values that $\tau_A$ could take, we distinguish three distinct cases:
\begin{enumerate}
\item $\tau_A < x$
\item $\tau_A > h_a$
\item $ x \leq \tau_A \leq h_a$
\end{enumerate}
We know by definition that the minimum percent of honest stake that we can tolerate is $x$. In the first case, we allow the activation of changes to take place before this minimum of honest stake has the chance to upgrade. Thus, we risk to activate too-early, we partition the honest stake in two and therefore cause a chain split \say{by accident}. In the second case, we impose an adoption threshold $\tau_A$ that is greater than the actual honest stake. Therefore, in order to activate, we need to wait not only for the honest stake to upgrade but also for some part of the adversary stake, which of course may refuse to do. Therefore, by activating too-late, we give the opportunity to the adversary to block the activation forever.

Finally, in the third case, we do not have the risk of the activation blocking by an adversary, because we do not depend on the adversary's signals to meet the $\tau_A$ threshold. We depend only on honest stake and we have assumed from the beginning that all honest stake will eventually upgrade (and thus signal). However, we still run the risk to activate too-early and cause a chain split, but this time not \say{by accident}, but as an attack. Let us assume that, at a specific run, we achieve to meet the stake threshold $\tau_A$ of signals. This stake, in general, will comprise some honest part and some adversary part, $\tau_A = S_{honest} + S_{adversary}$. Clearly, in order to be safe from an accidental chain split, we need $S_{honest} \geq x$. This way, the \emph{longest chain rule} will ensure that the upgraded honest stake will produce the prevailing chain of blocks. So in order to be safe, as a minimum we need $S_{honest} = x$. In conclusion, in order to avoid both of the two problems described above, the adoption threshold $\tau_A$ should take values in the range $x \leq \tau_A \leq h_a$, with the assumption that the honest stake that will signal, when the threshold is met, will be at least $x$ ($S_{honest} \geq x$) , i.e., at least equal to the honest stake threshold of the consensus protocol. 

The possible attack that we mentioned above in the case where $x \leq \tau_A \leq h_a$ is the adversary to hurry to signal for a software update, so that the threshold $\tau_A$ is met, without (at least) $x$ honest stake to have enough time to complete the upgrade (i.e., $S_{honest} < x$). Thus a too-early activation will take place and the honest stake will be partitioned for some time, risking a chain split and also running the consensus protocol without the $x$ honest stake assumption.

Intuitively, in order to prevent this type of attack, we need to give more time to honest stake to upgrade, since we have assumed that all honest stake will eventually upgrade. Therefore a high value of $\tau_A$ would help towards this end. However, we have seen from the analysis above that if we set the adoption threshold too high, we risk from the activation blocking attack. Therefore, for difficult to deploy software updates, or hard fork type of changes, our update policy, would ideally adjust the adoption threshold close to $h_a$ (i.e., to the actual percent of honest stake). Of course, we do not know $h_a$ and thus it is not easy to use the adoption threshold for this purpose. We need a way to delay the activation of changes even though the $\tau_A$ threshold has been met. This is exactly the topic of the next subsection.

%Then, the inequity $ x \leq \tau_A \leq h_a$ can be expressed as $ x \leq S_{honest} + S_{adversary} \leq h_a$, which can be expressed as $x \leq x + S_{adversary} \leq h_a$. The left part holds obviously and therefore we need $x + S_{adversary} \leq h_a$, which can be expressed as $S_{adversary} \leq h_a - x$. Therefore, the condition that we need, in order to ensure that the adoption threshold $\tau_A$ does not risk a safety nor a security problem is:
%$$
%x \leq \tau_A \leq h_a \quad \textrm{(1)} \quad \land \quad   S_{adversary} \leq h_a - x \quad \textrm{(2)}
%$$
%The (2) inequality means that the adversary stake who has signaled must be below the gap between actual honest stake $h_a$ and minimum honest stake $x$. If we assume that $h_a = x$ then the above two conditions become:
%$$
%\tau_A = x \quad \textrm{(3)} \quad \land \quad   S_{adversary} = 0 \quad \textrm{(4)}
%$$


%\paragraph{Activation Phase} 
%the activation phase is not a re-approval phase, it is just there to guard against chain split.s
%It is not relevant to UPs that dont impact consensus (adoption threshold = 0\%)
%For those that do impact, adoption threshold = the honest stake threshold assumption of the consensus protocol. We make the assumption that if a UP is approved then all honest stake will eventually upgrade.

\subsubsection{The Activation Lag}
To mitigate the risk of a too-early activation, either \say{by-accident} or by-attack, we propose the concept of the \emph{activation lag}. The activation lag is a metadata-driven artificial delay, which is imposed on the activation of a change, in order to give time to the honest stake to upgrade and thus protect against a too-early activation. In other words, even though  the adoption threshold of signals might be met, the activation of the changes does not take place, it is stalled, until the activation lag period passes.

The activation lag is determined by two things: a) the deployment complexity of a software update and b) the type of the consensus protocol change (soft/hard fork\footnote{Velvet forks \cite{velvet} do not impact the consensus by causing forks and thus can activate immediately after an upgrade.}).
For example, there might be software updates that entail a very complex deployment process; one that even maybe a hardware upgrade is required before the software upgrade. In such a case, the activation lag must ensure plenty of time to the stakeholders to upgrade. Similarly, a hard fork type of change, should trigger a greater activation lag than that of a soft fork type of change. 

As we have seen from the metadata subsection, we propose both of these two important aspects to be recorded as essential properties of a software update's metadata. This information will drive the choice of the length of the activation lag and enable a metadata-driven update policy.


%\subsection{The Activation Phase Revisited}


\subsection{Update Constraints}
With the term \emph{update constraints}, we mean all the prerequisites of a software update for a successful deployment and in particular those that deal with other software updates. 
More specifically, we are interested in \emph{dependencies} with other software updates and \emph{conflicts} with other software updates. 

These constraints should be clearly stated, if possible, even from the ideation phase, where the software update is just a SIP, but certainly, they must be declared in the metadata part of a UP. Essentially, an update  constraint is a predicate, which evolves, as the software update matures from an SIP to a UP,  and which is evaluated in three phases: a) at the ideation, b)the approval and c) at the activation phase. This evaluation acts as a filter, to protect from software updates that should not be activated due to missing prerequisites, or due to conflicts with other software updates.

\subsubsection{Dependencies}
Commonly, software dependencies are expressed in terms of version requirements. For example, a software update defines in its metadata, on which versions of the software it is applicable and therefore anyone who downloads it, knows if it can be applied, or not. In concept, the proposed method is not far from this idea.

Each instance of the blockchain system is uniquely characterized by a version number. This version number will be based on some versioning scheme, which can be as elaborate as required by the software needs. Every time a software update is activated in the activation phase, the current version of the system changes to some new value. The value that the version will reach after the activation of the software update is also recorded in the update metadata. The current version of the system is also evident in each generated block. Also, as we have seen in the activation phase and the signaling mechanism, blocks are also marked with the to-be version of the system, after the upgrade takes effect. Finally, for each software update the base version(s) should be clearly stated in the metadata. These are the versions that this update can be applied to, with no problem. 

All in all, for each software update, we know on what versions it is applicable, to what version it will take the system to and what is the current version of the system. These are sufficient information (even in the centralized setting) for any approver to judge for the applicability of a software update with respect to software dependencies.

%
%How do we impose adherence to update dependencies, so that the system reaches a consistent state? 
%\nnote{Maybe we could use the version\_from field from the metadata for this purpose. An Update Proposal cannot be applied if the version requirement that it poses (version\_from) is not the current version}
%

\subsubsection{Conflict Resolution}
%How do we ensure that multiple concurrent requests for updates are handled simultaneously in a way that
%\begin{itemize}
%\item conflicts are resolved and the adopted updates are consistent, 
%\item so that no contradictory updates are to be deployed at the same time
%\item Community splits over controversial updates are avoided
%\end{itemize}
Conflicts of software updates in general have to do with conflicts of different versions of the source code. It is typical when a new version of a software is merged on some other version (main branch) to have a conflict at the source code level. This is typically reviewed and maybe resolved by the code maintainer. In the decentralized setting these type of conflicts are resolved in a similar manner. We assume that each approver in the approval phase, maintains a local code repository and tries to verify the (metadata-declared) applicability of a software update on the indicated version of the code, by a simple merge operation. If this merge raises a conflict, then the approver must reject the software update.

More specifically in blockchain systems, there is another type of conflicts that is of particular importance. These are the conflicts that are related to the consensus protocol parameters. For example, a software update increases the maximum transaction size and another software update decreases it. These two SUs are clearly in conflict. However, they are in conflict only if the are both \emph{open} at the same time, otherwise they are just a valid sequence of changes of the system. We call a software update as \emph{open}, if it is in some of the four phases comprising the lifecycle of a SU, namely: the ideation, the implementation, the approval, or the activation phase. If there are two (or more) SUs, in any of these phases and they impact the same protocol parameters (which is a metadata-recorded piece of information), then we have a conflict and must perform some resolution action. The resolution action might vary based on the needs of the system and the context of the software update. One possible action would be to reject all conflicting SUs. Another possible action would be to reject all conflicting SUs, except one (if there is only one), at the latest phase. All these decisions, have to do with the criticality of a software update, the type of the software update and other parameters, which should all be defined in the software update metadata, in order to help to define some update policy. So we see once more, how it is possible with an appropriate definition of software update metadata, to drive also the update constraint evaluation process.


\subsection{Rollbacks}
How can we smoothly rollback an update, in the case of a problem?


%\subsubsection*{Update Policies per type of updates -- old stuff}
%An \emph{Update Policy} is defined as the pair \emph{(Speed of Activation, Method of Deployment)}. We need to differentiate the speed of deployment and the method of deployment based on: a) the type of change (bug -fix or change request), b) the part of the system that is affected by the change (consensus rules impact, or only software impact), c) the urgency of the change (severity level)
%and d) soft vs. hard forks
%
%\todo{Nikos: We have 3 levels of speed: high, medium and low. What are our different deployment methods?}
%
%\begin{itemize}
%\item How do we discriminate between different types of updates (e.g., software vs. protocol, bug-fix vs. change request)
%\item We need to provide a different deployment path for each Type of Update. For example, critical hot-fixes might need to bypass some of the governance steps.
%\item We need to incorporate into our update logic the notions of bug Severity and ''Required Speed of Deployment''
%\end{itemize}
%
%Also, deployment/activation time must vary according to the type of change. We see the following ''change categories'':
%\begin{itemize}
%\item Bug-fix vs. Change Request
%\item Consensus Protocol impact vs. No Impact
%\item High severity vs. Low severity
%\end{itemize}

