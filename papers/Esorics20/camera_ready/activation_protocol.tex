\section{Our constructions}
In this section we propose two main approaches to turn a ledger $\ledger_1$ into an updatable ledger $\ledgerup$. That is, we show how to make $\ledger_1$ able to self-update to the code of a new ledger $\ledger_2$.
The first approach proposed requires $\ledger_1$ and $\ledger_2$ to be the same (i.e.,  they use the same consensus rules) but might have a different block structure. The advantage in this approach is that we get a very simple updatable ledger, that does not decrease the throughput of  $\ledgerup$ during the update and does not require all the honest parties to be online during the update\footnote{We also show that we can relax the requirement on the honest parties being online during the update for the case of PoW ledgers.}. 
The second approach requires fewer similarities between the two ledgers, but it is proven secure only in the semi-online. We also show that we can relax the requirement on the honest parties being online during the update by relying on a \emph{2-for-1 mining} approach (more details are provided in the end of Sec.~\ref{se:first}).

We now provide a detailed description of our approaches and formally prove their security.




\subsection{First approach.}
In this section we consider a simplified scenario where the two ledgers, $\ledger_1$ and $\ledger_2$, 
are the same except for the block format (i.e., $\ledger_1$ and $\ledger_2$ might have a different block size).
Moreover, we assume that a block valid for $\ledger_1$ is valid for $\ledger_2$ as well (but the viceversa does not necessarily hold). Formally, this means that if the block validation algorithm  $\isvalidb_1$ of $\ledger_1$
outputs $1$ on some input $B$, then also the block validation algorithm $\isvalidb_2$ of $\ledger_2$ outputs $1$ (see Sec.~\ref{se:ledgermodel} for more details).
We now prove the following theorem


\begin{theorem}
If $\ledger_1$ and $\ledger_2$ are secure ledgers with block validation rules respectively $\isvalidb_1$ and $\isvalidb_2$ such that: 
\begin{enumerate}
	\item  $\ledger_1$ and $\ledger_2$ are the same except with respect to the block validation rules;
	\item for every block $B$ such that if $\isvalidb_1(B)=1$ then $\isvalidb_2(B)=1$,
	\item $\ledger_1$ (resp. $\ledger_2$)  has common-prefix parameter $k$, chain-growth parameter  $(\cg, s)$ and assumption $\asmp_1$ (resp. $\asmp_2$) with $\asmp_1=\asmp_2$,
	
\end{enumerate}
then there exists an updatable ledger $\ledgerup$ with update parameter  $\Delta := (k+1) \cg^{-1} + s$.
\end{theorem}



\begin{proof}
We assume that enough parties have received the command $(\activate, \ledger_2)$ such that
$\asmp_2$ holds and denote the time when this happen with $T_0$.
Our updatable ledger $\ledgerup$ works as follows.

Each party $P_i\in\activep$ does the following steps.
\begin{enumerate}
	\item Use $\isvalidb_2$ as a block validation algorithm.
	\item Create and post a transaction that contains an \emph{activation flag}. 
	\item Let $\aflag$ be the index of the block that will contain the first transaction with an activation flag.
	\item Let $j:=\aflag+k+1$, run $\ledger_1$ and when the $j$-th block $B^i_j$ becomes part of $\check \state_1^{P_i}[\tau_i]$ for some $\tau_i \geq T_0$ start extending $B^i_{j}$ using the rules of $\ledger_2$ instead of the rules of $\ledger_1$ (we recall that a valid block for $\ledger_1$ is also a valid block for $\ledger_2$)
\end{enumerate}


We provide a pictorial description of what happens to the ledger state during the update in Fig.~\ref{fig:sketch2}.
We note that two honest parties $P_1$ and $P_2$ might have different $\check \state_1^{P_1}[\tau]$ and  $\check \state_1^{P_2}[\tau]$ at any time $\tau$. The Fig.~\ref{fig:sketch2}
describe the scenario where $P_1$ might start to run $\ledger_2$ starting from an 
unstable block (i.e. a block of  $\check \state_1^{P_1}[\tau]$ with $\tau\geq
T_0+s$) which is different from the block that $P_2$ is using. However, after sufficiently many
 rounds (at some round  $\tau'\leq T_0+s+(k_1+1)\cg_1^{-1}$ to be precise) $P_1$ and $P_2$ will agree on what is the last block of $\ledger_1$ and what is the first bock of $\ledger_2$.

 
\begin{figure}
\setlength{\fboxsep}{12pt}
\begin{center}
\begin{boxedminipage}{12cm}

\begin{center}

\psscalebox{1.0 1.0} % Change this value to rescale the drawing.
{
\begin{pspicture}(0,-3.8829687)(13.72,3.8829687)
\definecolor{colour0}{rgb}{0.8,0.8,0.8}
\definecolor{colour1}{rgb}{0.4,0.4,0.4}
\psframe[linecolor=black, linewidth=0.04, fillstyle=solid,fillcolor=colour0, dimen=outer](2.0,1.7929688)(0.7,1.0929687)
\psline[linecolor=black, linewidth=0.04, linestyle=dotted, dotsep=0.10583334cm](2.04,1.4929688)(2.2,1.4929688)
\psline[linecolor=black, linewidth=0.04](2.36,1.4929688)(2.6,1.4929688)
\psframe[linecolor=black, linewidth=0.04, fillstyle=solid,fillcolor=colour0, dimen=outer](3.7,1.7929688)(2.4,1.0929687)
\rput[bl](2.8,1.2929688){flag}
\rput[bl](2.7,-0.70703125){$\aflag$}
\psline[linecolor=black, linewidth=0.04](3.7,1.4929688)(4.0333333,1.4929688)(4.0333333,1.4929688)
\psline[linecolor=black, linewidth=0.04, linestyle=dotted, dotsep=0.10583334cm](4.0571427,1.4929688)(4.2,1.4929688)
\psline[linecolor=black, linewidth=0.04](4.5333333,0.39296874)(4.8,0.39296874)
\psframe[linecolor=black, linewidth=0.04, linestyle=dashed, dash=0.17638889cm 0.10583334cm, fillstyle=solid,fillcolor=colour1, dimen=outer](6.1,0.6929687)(4.8,-0.00703125)
\psline[linecolor=black, linewidth=0.04](6.1,0.39296874)(6.5,0.39296874)
\psframe[linecolor=black, linewidth=0.04, fillstyle=solid,fillcolor=colour0, dimen=outer](1.3,-1.6070312)(0.6,-2.0070312)
\rput[bl](1.6,-1.9070313){$\ledger_1$ block}
\rput[bl](5.0,-0.70703125){$\aflag+k+1$}
\rput[bl](8.9,-0.70703125){$\aflag+k$}
\rput[bl](1.9,3.5929687){$T_0$}
\rput[bl](2.8,3.5929687){$T_0+s$}
\rput[bl](8.3,3.5929687){$T_0+s_1+(k_1+1)\cg_1^{-1}$}
\rput[bl](0.2,3.5929687){{\bf Round}}
\rput[bl](0.0,-0.70703125){{\bf Block index}}
\psframe[linecolor=black, linewidth=0.04, linestyle=dashed, dash=0.17638889cm 0.10583334cm, fillstyle=solid,fillcolor=colour1, dimen=outer](7.8,0.6929687)(6.5,-0.00703125)
\psframe[linecolor=black, linewidth=0.04, linestyle=dashed, dash=0.17638889cm 0.10583334cm, fillstyle=solid,fillcolor=colour0, dimen=outer](1.3,-2.2070312)(0.6,-2.6070313)
\rput[bl](1.7,-2.6070313){Unstable $\ledger_1$ block}
\psline[linecolor=black, linewidth=0.04](4.5333333,2.9929688)(4.8,2.9929688)
\psframe[linecolor=black, linewidth=0.04, linestyle=dashed, dash=0.17638889cm 0.10583334cm, fillstyle=solid,fillcolor=colour0, dimen=outer](6.1,3.2929688)(4.8,2.5929687)
\psline[linecolor=black, linewidth=0.04](6.1,2.9929688)(6.5,2.9929688)
\psframe[linecolor=black, linewidth=0.04, linestyle=dashed, dash=0.17638889cm 0.10583334cm, fillstyle=solid,fillcolor=colour1, dimen=outer](7.8,3.2929688)(6.5,2.5929687)
\psline[linecolor=black, linewidth=0.04, linestyle=dotted, dotsep=0.10583334cm](7.8,2.9929688)(8.2,2.9929688)
\psframe[linecolor=black, linewidth=0.04, fillstyle=solid,fillcolor=colour1, dimen=outer](1.3,-2.8070312)(0.6,-3.2070312)
\rput[bl](1.7,-3.2070312){$\ledger_2$ block}
\psline[linecolor=black, linewidth=0.04, linestyle=dotted, dotsep=0.10583334cm](7.8,0.39296874)(8.3,0.39296874)(8.3,0.39296874)
\rput[bl](6.5,1.9929688){$\aflag+k+1$}
\psline[linecolor=black, linewidth=0.04, linestyle=dotted, dotsep=0.10583334cm](4.6,2.9929688)(4.3,2.9929688)
\psline[linecolor=black, linewidth=0.04, linestyle=dotted, dotsep=0.10583334cm](4.6,0.39296874)(4.3,0.39296874)
\psframe[linecolor=black, linewidth=0.04, fillstyle=solid,fillcolor=colour0, dimen=outer](9.8,1.7929688)(8.5,1.0929687)
\psframe[linecolor=black, linewidth=0.04, fillstyle=solid,fillcolor=colour1, dimen=outer](11.4,1.7929688)(10.1,1.0929687)
\rput[bl](10.0,-0.70703125){$\aflag+k+1$}
\psline[linecolor=black, linewidth=0.04](2.2,3.3929687)(2.0,2.9929688)(2.3,2.5929687)(2.0,2.1929688)(2.3,1.7929688)(2.0,1.2929688)(2.3,0.8929688)(2.3,0.8929688)
\psline[linecolor=black, linewidth=0.04, linestyle=dotted, dotsep=0.10583334cm](8.5,1.4929688)(8.2,1.4929688)
\psline[linecolor=black, linewidth=0.04](9.8,1.4929688)(10.1,1.4929688)
\psline[linecolor=black, linewidth=0.04](10.8,3.3929687)(10.6,2.9929688)(10.8,2.6929688)(10.6,2.2929688)(10.8,1.8929688)(10.8,1.8929688)
\psframe[linecolor=black, linewidth=0.04, linestyle=dashed, dash=0.17638889cm 0.10583334cm, fillstyle=solid,fillcolor=colour1, dimen=outer](1.3,-3.4070313)(0.6,-3.8070312)
\rput[bl](1.7,-3.8070312){Unstable $\ledger_2$ block}
\end{pspicture}
}


\end{center}

\end{boxedminipage}
\end{center}
\caption{Transition from $\ledger_1$ to $\ledger_2$. Note that different honest parties might have different views (i.e., forks) of the unstable part of the chain which have also different lengths.} 
\label{fig:sketch2}
\end{figure}

To complete the proof we need to show that $\ledger_2$ enjoys consistency and liveness
and that the state $\ledger_1$ at some time $\tau\in [T_0,T_1]$ is a prefix of $\ledger_2$'s state. 


\import{./}{proof}

\end{proof}



We finally note that this protocol does not put any restriction on whether an honest party needs to be online or not during an update given that $\ledger_1$ and $\ledger_2$ have the same chain selection rule (only the block selection rule is different).
One practical advantage of our approach is that if $\ledger_1$ (and $\ledger_2$) allows bootstrapping from the genesis block (like in~\cite{CCS:BGKRZ18}) so does our updatable ledger.




\subsection{Second Approach}\label{se:first}
Before providing our construction we introduce the notion of \emph{\gencompatible} ledgers. We say that
two ledgers $\ledger_1$ and $\ledger_2$ are \gencompatible\ if a block of 
$\ledger_1$ can be turned into a valid candidate genesis block for $\ledger_2$. We now propose a formal definition.


\begin{definition}
Let $\ledger_1$ and $\ledger_2$ be two secure ledgers where $\fgenesis$ is the genesis functionality 
of $\ledger_2$ parameterized by the algorithm $\genesis()$ (see Fig.~\ref{fig:genesis}).

We say that $\ledger_1$ is \gencompatible\ with $\ledger_2$ if there exists a deterministic polynomial time
algorithm $\translate$ that, on input a valid block $B$ of $\ledger_1$ 
outputs a valid genesis block $\tilde B$ for $\ledger_2$. Moreover, the output of $\translate$ is identically distributed to the output of the procedure $\genesis()$.
\end{definition} 



We note that $\translate$ could be a very simple protocol. For example, if we consider two PoW ledgers 
that use the same puzzles, then $\ledger_1$ is 
\gencompatible\ with $\ledger_2$ since the $\translate$ can simply take a block of $\ledger_1$ and use it as
a candidate genesis block for $\ledger_2$.
We note that the definition of genesis-compatibility only tells that it is possible to generate a genesis block for $\ledger_2$ with a valid structure. That is, it does not imply that $\ledger_2$ can be securely run using any genesis block generated using $\translate$ as, for example, using an old block of $\ledger_1$ could give an advantage to the adversary over the honest parties.  More details follow.

We now propose our first compiler that turns a ledger $\ledger_1$ that is \gencompatible\ with $\ledger_2$,
into an updatable ledger.
At a very high level our approach is the following. We use $\ledger_1$ to realize the genesis functionality 
of $\ledger_2$, and then we use the output of the genesis functionality to execute $\ledger_2$.
We note that it is easy to create a candidate genesis block from $\ledger_1$ because it is \gencompatible\
with $\ledger_2$.
To complete the description of our compiler, we need to specify what block of $\ledger_1$ will be chosen, and
argue that this process is indeed sufficient to realize the genesis functionality for $\ledger_2$.
In our approach the parties that are running $\ledger_1$ agree on the index $j$ of a block that will be used
as a genesis block (this block can be decided using the consensus algorithm of $\ledger_1$, more details will be provided). 
When the block of position $j$, that we denote with $B_j$, becomes stable for all the honest parties that decided to update, then these parties use $\translate$ to turn $B_j$ into a genesis block for $\ledger_2$ thus obtaining
$\bgen$. At this point $\bgen$ is used to run $\ledger_2$ and $\ledger_1$ can be abandoned.

Even though the above approach seems to work, there are many subtleties. The first is that the adversary might 
be able to see the block $B_j$ before any other honest parties do, and therefore he can take an advantage in the generation
of the blocks of $\ledger_2$. The second issue is that the adversary might influence the choice of the block
that will appear in position $j$. Indeed, we do not know how the consensus algorithm of $\ledger_1$ works and what is the power of the adversary in biasing the content of $B_j$. 
We denote with ${\delay}'$ the upper bound on the number of rounds that pass between the time at which 
the adversary can see a candidate
block for $\ledger_1$ for a position $j$, and the time at which all the honest parties see $B_j$ as part of the stable chain. We refer to this parameter ${\delay}'$ as the \emph{prediction} parameter. 
We also denote with ${\maxblocks}'$ the upper bound on the number of valid chains that are broadcasted on the network that contain a block in position $j$ and refer to this parameter as \emph{maximum forks} parameter.

Coming back to our protocol, we note that if the genesis functionality of $\ledger_2$ is parameterized
with $\delay={\delay}'$ and $\maxblocks={\maxblocks}'$ then we can prove that the solution we proposed works.

We are now ready to state formally our theorem and prove it.

\begin{theorem}
If $\ledger_1$ and $\ledger_2$ are secure ledgers and:
\begin{enumerate}
	\item $\ledger_1$  has common-prefix parameter $k_1$, chain-growth parameter  $(\cg_1, s_1)$ and
	assumption $\asmp_1$;
	\item $\ledger_2$  has common-prefix parameter $k_2$, chain-growth parameter  $(\cg_2, s_2)$ and
	assumption $\asmp_2$;
	\item the prediction parameter of $\ledger_1$ is ${\delay}'$ and the maximum forks parameter is ${\maxblocks}'$;
	\item the genesis functionality $\fgenesis$ of $\ledger_2$ is parametrized by $\delay={\delay}'$ and 
	$\maxblocks={\maxblocks}'$;
	\item $\ledger_1$ is genesis-compatible with $\ledger_2$.
\end{enumerate}

then there exists an updatable ledger $\ledgerup$ with update parameter $\Delta:=2k_1 \cg_1^{-1}+s_1$ in the
semi-online setting.

\end{theorem}




\begin{proof}
We start the proof by describing how formally our protocol works.
Let $T_0$ be such that $\asmp_2$ holds. At time $T_0$ each party in $P_i\in \activep$ does the following steps.

\begin{enumerate}
	\item Create and post a transaction that contains an \emph{activation flag}, let $\aflag$ be the index of the block that will contain the first transaction with an activation flag (note that there might be more than one of such a transactions).
	\item Keep running $\ledger_1$ until the block with index  $j=\aflag+k_1$ becomes stable (i.e.,becomes part of $\state_1^{P}[\tau]$ for all $P\in\activep$ for some $\tau \geq T_0$) and stop issuing transaction for $\ledger_1$ (if any).
	\item When the $j$-th block $B_{j}$  becomes stable then stop running $\ledger_1$ and start running 
	$\ledger_2$ using $\bgen\leftarrow\translate(B_{j})$ as the genesis block.
	
\end{enumerate}


We provide a pictorial description of what happens to the ledger state during the update in Fig.~\ref{fig:sketch}.
The activation flag is used by the honest parties to reach an agreement on what it will be the index of the block used as a genesis block. We note that the blocks of $\ledger_1$ that extend $B_j$ might be unstable, moreover after the update has been 
completed the parties in $\activep$ will ignore the blocks of $\ledger_1$ that extend $B_j$ (since after the update all the parties in $\activep$ will be using the rules $\ledger_2$, hence its chain selection rule).  The reason why the parties in $\activep$ will stop issuing transactions for $\ledger_1$ is that these transactions might be included in blocks that extend $B_j$, which will be ignored after $T_0+\Delta$ rounds. This clearly affects the throughput of the ledger in the interval $[T_0+k_1 \cg_1^{-1}+s_1, T_0+2k_1 \cg_1^{-1}+s_1]$ (Fig.~\ref{fig:sketch}).

We now continue with the proof.
Let $T_0$ be the time at which we know that  $\activep$ is such that $\asmp_2$ holds.
In the worst case, the time required for an honest party to post a transaction that 
contains the activation flag takes time $s_1$ rounds ($s_1$ comes from
the liveness of $\ledger_1$).
The number of rounds required for $j$ to be stable in the view of all the honest parties is 
 $2k_1\cg_1^{-1}$ rounds. This is because to generate the block $B_j$ are required at least $k_1\cg_1^{-1}$ 
 rounds, and $B_j$ has to be extended with at least $k_1$ blocks to be part of all the honest parties view (and this takes additional $k_1\cg_1^{-1}$ rounds) 
  Hence, the time required to complete the update is $\Delta=2k_1 \cg_1^{-1}+s_1$.
 Once the block $B_j$ becomes stable, the parties in $\activep$ can start running $\ledger_2$, and 
 we are guaranteed that $\ledger_2$ enjoys liveness and consistency because the genesis block
 for $\ledger_2$ is created accordingly to $\fgenesis$ and by assumption $\asmp_2$ holds.
  Therefore, everything that appears before $\bgen$ is preserved due to the consistency of 
 $\ledger_2$. 
 
 We refer to the state of $\ledger_1$ before $\bgen$ as $\tilde \state_1$, and to the state
 of the ledger after the update as $\tilde \state_1||\state_2$.
 We finally note that we guarantee no security for the honest parties that were not online during the update.
 The reason is that after $T_1$ the honest parties abandon $\ledger_1$ and the adversary could compromise it.
 For example, an adversary could potentially keep extending $\state_1$ after the block $j$, and 
 create a very long chain, even longer that $\tilde \state_1||\state_2$. Hence, if the chain selection rule of
 $\ledger_1$ prescribes to take the longest chain, then a party that comes online at time $T_1$ might take
 the chain $\state_1$ (which is compromised). 
 
\end{proof}


We remark that our construction requires the parties to generate empty blocks for $\ledger_1$ from block index $j+1$ and until block $B_j$ becomes stable. This is required as the honest parties, after the update completes, will ignore any block generated using the rules of $\ledger_1$ that comes after 
$B_j$. 



\begin{figure}
\setlength{\fboxsep}{12pt}
\begin{center}
\begin{boxedminipage}{13cm}

\begin{center}
% \usepackage[usenames,dvipsnames]{pstricks}
% \usepackage{epsfig}
% \usepackage{pst-grad} % For gradients
% \usepackage{pst-plot} % For axes
% \usepackage[space]{grffile} % For spaces in paths
% \usepackage{etoolbox} % For spaces in paths
% \makeatletter % For spaces in paths
% \patchcmd\Gread@eps{\@inputcheck#1 }{\@inputcheck"#1"\relax}{}{}
% \makeatother
% 
\psscalebox{1.0 1.0} % Change this value to rescale the drawing.
{
\begin{pspicture}(0,-2.6829689)(14.2,2.6829689)
\definecolor{colour0}{rgb}{0.8,0.8,0.8}
\definecolor{colour1}{rgb}{0.4,0.4,0.4}
\psframe[linecolor=black, linewidth=0.04, fillstyle=solid,fillcolor=colour0, dimen=outer](1.7,2.0929687)(0.4,1.3929688)
\psline[linecolor=black, linewidth=0.04](1.7,1.7929688)(1.94,1.7929688)(1.94,1.7929688)
\psline[linecolor=black, linewidth=0.04, linestyle=dotted, dotsep=0.10583334cm](1.94,1.7929688)(2.3,1.7929688)
\psline[linecolor=black, linewidth=0.04](2.36,1.7929688)(2.6,1.7929688)
\psframe[linecolor=black, linewidth=0.04, fillstyle=solid,fillcolor=colour0, dimen=outer](3.9,2.0929687)(2.6,1.3929688)
\rput[bl](3.0,1.5929687){flag}
\rput[bl](2.9,0.09296875){$\aflag$}
\psline[linecolor=black, linewidth=0.04](3.9,1.7929688)(4.1666665,1.7929688)(4.1666665,1.7929688)
\psline[linecolor=black, linewidth=0.04, linestyle=dotted, dotsep=0.10583334cm](4.1666665,1.7929688)(4.5666666,1.7929688)
\psline[linecolor=black, linewidth=0.04](4.633333,1.7929688)(4.9,1.7929688)
\psframe[linecolor=black, linewidth=0.04, fillstyle=solid,fillcolor=colour0, dimen=outer](6.2,2.0929687)(4.9,1.3929688)
\psline[linecolor=black, linewidth=0.04](6.2,1.7929688)(6.6,1.7929688)
\psframe[linecolor=black, linewidth=0.04, linestyle=dashed, dash=0.17638889cm 0.10583334cm, fillstyle=solid,fillcolor=colour0, dimen=outer](7.9,2.0929687)(6.6,1.3929688)
\psline[linecolor=black, linewidth=0.04](7.9,1.7929688)(8.3,1.7929688)(8.3,1.7929688)
\psline[linecolor=black, linewidth=0.04, linestyle=dotted, dotsep=0.10583334cm](8.3,1.7929688)(8.9,1.7929688)
\psline[linecolor=black, linewidth=0.04](9.0,1.7929688)(9.4,1.7929688)
\psframe[linecolor=black, linewidth=0.04, linestyle=dashed, dash=0.17638889cm 0.10583334cm, fillstyle=solid,fillcolor=colour0, dimen=outer](10.7,2.0929687)(9.4,1.3929688)
\psline[linecolor=black, linewidth=0.04](5.6,1.3929688)(5.6,0.8929688)(9.4,0.8929688)(9.4,0.8929688)
\psframe[linecolor=black, linewidth=0.04, fillstyle=solid,fillcolor=colour1, dimen=outer](10.7,1.1929687)(9.4,0.49296874)
\psline[linecolor=black, linewidth=0.04, linestyle=dotted, dotsep=0.10583334cm](10.7,0.8929688)(11.4,0.8929688)(11.4,0.8929688)
\psframe[linecolor=black, linewidth=0.04, fillstyle=solid,fillcolor=colour0, dimen=outer](1.1,-1.0070312)(0.4,-1.4070313)
\psframe[linecolor=black, linewidth=0.04, linestyle=dashed, dash=0.17638889cm 0.10583334cm, fillstyle=solid,fillcolor=colour0, dimen=outer](1.1,-1.6070312)(0.4,-2.0070312)
\psframe[linecolor=black, linewidth=0.04, fillstyle=solid,fillcolor=colour1, dimen=outer](1.1,-2.2070312)(0.4,-2.6070313)
\rput[bl](1.4,-1.2070312){$\ledger_1$ block}
\rput[bl](1.5,-1.9070313){Unstable $\ledger_1$ block}
\rput[bl](1.5,-2.6070313){$\ledger_2$ block}
\rput[bl](5.3,0.09296875){$j=\aflag+k_1$}
\rput[bl](9.7,0.09296875){$\aflag+2k_1$}
\rput[bl](2.0,2.3929687){$T_0$}
\rput[bl](2.9,2.3929687){$T_0+s_1$}
\rput[bl](4.9,2.3929687){$T_0+k_1\cg_1^{-1}+s_1$}
\rput[bl](9.3,2.3929687){$T_0+2k_1\cg_1^{-1}+s_1$}
\rput[bl](0.0,2.3929687){{\bf Round}}
\rput[bl](0.0,0.19296876){{\bf Block index}}
\end{pspicture}
}




\end{center}

\end{boxedminipage}
\end{center}
\caption{Transition from $\ledger_1$ to $\ledger_2$. Note that the empty blocks of $\ledger_1$ might be non-stable.} 
\label{fig:sketch}
\end{figure}


\subsubsection{Practical implications.}

The updatable ledger that we have described can be updated to any ledger $\ledger_2$ under the condition 
that the genesis functionality of $\ledger_2$ tolerates an adversary that can see the genesis block
$\delay$ rounds before the honest parties and decide the genesis block among a set of $\maxblocks$ candidate 
genesis blocks. This requirement might look strong, but we note that the problem of constructing a ledger 
that is secure in such a scenario is simpler than the problem of constructing a ledger that supports temporary dishonest majority~\cite{FC:AKWW19}. A ledger with security assumption $\asmp$ that tolerates temporary dishonest majority is such that its security properties (liveness and consistency) become valid again 
when $\asmp[\tau_1]=1$, even if $\asmp[\tau']=0$ for all $\tau'\in[\tau_0, \tau_1-\delta]$
for some $\tau_0,\tau_1,\delta \in\mathbb{N}$ such that  $\tau_1-\delta\geq \tau_0$. That is,
the ledger become secure again when there is honest majority (i.e., $\asmp$ holds) even if there was an interval of time when there was no honest majority (i.e., $\asmp$ did not hold).
Therefore, if we consider the extreme case where $\tau_0=0$, we can assume without loss of generality 
that the ledger admits a genesis functionality parametrized by $\delay=\delta$, and by $\maxblocks$
that depends on the upper bound on the number of forks that the adversary can create. 
Hence, there are already ledgers that might fit our requirements for $\ledger_2$, and all the advancement in the 
research that concerns the security of ledgers in the case of temporary dishonest majority can be used 
to construct good candidates of updated ledgers ($\ledger_2$) for existing ledgers ($\ledger_1$) that can be used in our compiler.

\subsubsection{Security for Offline Parties}
\newcommand{\concat}{\,\|\,}

Our security notion above is ensured for parties that are online during the upgrade process. Clearly it
is necessary that the majority of the population's consensus-maintaining parties are honest and
online, as the honest majority assumption mandates. Nevertheless, practical blockchain systems often have
a large number of \emph{consumer} parties by count who have a very small contribution to the total computational power of
network, if at all, and are not significantly contributing to the maintenance of the consensus.
These nodes can be wallets and other clients who mainly consume, rather than maintain, the blockchain,
and are often offline for longer periods of time.
Regardless, these nodes constitute the economic majority of the nodes and we must ensure
they can also upgrade safely.

The critical situation arises when such a party goes offline prior to an upgrade, remains offline during
every phase of the upgrade, and comes online long after the rest of the population has successfully
upgraded. Before describing how to construct a protocol that can protect these parties, let us briefly observe
why an attack is easily possible by a minority adversary in a construction with no relevant protective mechanism.
Consider a situation where a hard-fork-style change takes place and that blocks mined by upgraded parties after
the upgrade are incompatible with blocks mined prior to the upgrade, i.e., after the upgrade, an unupgraded party
will not consider an upgraded block as valid and an upgraded party will not consider an unupgraded block as valid.
After the upgrade has been completed, the majority of the population will shift their mining power to mining new-style
blocks. The adversary can take advantage of this situation to \emph{ex post facto} attack the old system, which
now remains unprotected as no significant mining power remains to secure it. As such, she can break
the \emph{common prefix} property, rewrite history, and subvert the upgrade signaling mechanism itself. More
concretely, an adversary in this situation forks the old chain from the parent of the block in which upgrade
information appeared for the first time and continues mining a chain parallel to the one that yielded
the upgrade. As soon as that alternative history overtakes the old chain in terms of work, the adversary is
successful. Any offline party who wakes up afterwards will use the old-style consensus rules to choose
the blockchain and hence the upgrade will not appear in its view. The adversary has
succeeded in isolating the offline party from the rest of the network.

To rectify the above issue, a practical implementation of the protocol must leverage the mining power of the
upgraded population to maintain \emph{both} the new chain while at the same time securing the old chain. We propose a solution for the case where $\ledger_1$ and $\ledger_2$ are two proof-of-work (PoW) type of ledgers.
Our solution leverages on a variation of 2-for-1 mining~\cite{EC:GarKiaLeo15}.
An upgraded miner works as follows. They maintain the longest chain $C$ in view of the new protocol rules, but
also the longest chain $C'$ in the view of an unupgraded party. In case of hard fork, these two chains will
differ. When they are about to mine a new block on top of the upgraded chain,
they construct a new-style candidate block $b$ extending $C$ as usual. In addition, they also construct an empty (transactionless) old-style block $b'$ on top of the best unupgraded chain $C'$. In a commentary section of the old-style candidate block $b'$, such as the coinbase transaction, the miner
places the hash $H(b)$ of the new-style candidate block. The miner then attempts to find proof-of-work for the old-style block,
i.e., some nonce $\textsf{ctr}$ that satisfies the proof-of-work equation $H(b' \concat \textsf{ctr}) \leq T$ for the mining target $T$.
If such proof-of-work is found, then the block $b'$ is broadcast to the network and adopted as the tip of the longest unupgraded chain by
the rest of the (upgraded or unupgraded) miners. Note that this block is designed to be backwards-compatible in the sense that
it will be accepted by unupgraded miners even though they remain unaware of the upgrade. On the other hand, if the \emph{reverse proof-of-work equation} $H(b' \concat \textsf{ctr})^R \leq T$ is satisfied (where $H(\cdot)^R$ denotes the reversed bitstring of $H(\cdot)$), then $b'$ and the respective proof-of-work and blocks $b', b$ are
broadcast to the network. This time unupgraded miners will not consider this a valid block. However,
upgraded miners examine the validity of the block $b$ contained within the commentary section of $b'$ and check that the reverse proof-of-work equation is
satisfied. If so, they adopt the block $b$ as the next block in their upgraded blockchain. The above mechanism is the only mechanism
by which new-style blocks are accepted by upgraded honest miners.

The protocol just described has two advantages. Firstly, the upgraded honest miners make use of their mining power to contribute to the security
of both the old and the new-style chain simultaneously. Therefore, an adversary cannot attack the old chain \emph{ex post facto}.
Secondly, instead of \emph{dividing} their mining power between the two chains, the honest parties only use their mining power \emph{once} to mine on \emph{both} networks, because the hash function is only evaluated once. As such, the honest mining power is not diminished by the use of this mechanism. We observe that, in the Random Oracle model, the last bits of the hash output remain
uniformly distributed conditioned on the fact that the proof-of-work equation has a solution. Therefore, finding a solution of the
proof-of-work equation and finding a solution of the reverse proof-of-work equation are two independent events (they will occur simultaneously so rarely that the honest parties can ignore this possibility). Lastly, note that this scheme can be used repeatedly
when multiple upgrades have occurred on top of one another, simply by treating a portion of the bits of the hash as the significant
bits to test against the proof-of-work equation (e.g., for a second upgrade, the hash output can be split in three equal parts to be tested against the proof-of-work equation). This scheme therefore theoretically resolves the question of securing offline parties.
In practice, because the scheme adds significant implementation complexity, implementors may elect to maintain this backwards-compatibility
mechanism for a limited amount of time. In that case, parties who have remained offline longer than the backwards-compatibility
mechanism is maintained, will have no guarantees for security, similarly to a classical system whose long-term support window has expired.

The scheme requires the added complexity of mining two blocks simultaneously
only in the case of proof-of-work. This is due to the nature of proof-of-work and specifically the fact that each query counted
towards the proof-of-work quota can only be devoted to a specific message.
In proof-of-stake blockchains, the solution for maintaining the security of offline unupgraded parties is
the obvious one and allows for a much simpler implementation: We require upgraded parties to mint, alongside their
new-style blocks extending the longest upgraded chain and containing transactions,
also empty old-style blocks extending the longest unupgraded chain, to ensure the security of their unupgraded
counterparts.
