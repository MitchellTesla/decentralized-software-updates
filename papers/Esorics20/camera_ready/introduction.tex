\section{Introduction}

Most of the existing software requires to be updated (or replaced) at some point. Indeed, the most vital aspect for the sustainability of any software system is its ability to effectively and swiftly adapt to changes; one basic form of which are software updates. Therefore the adoption of software updates is at the heart of the lifecycle of any system, and blockchain systems are no exception. Software updates might be triggered by a plethora of different reasons: change requests, bug-fixes, security holes, new-feature requests, various optimizations, code refactoring etc.
More specifically, for blockchain systems, a typical source of change is the 
enhancements at the consensus protocol level. There might be changes to the 
values of specific parameters (e.g., the maximum block size, or the maximum 
transaction size etc.), changes to the validation rules at any level 
(transaction, block, or blockchain), or even changes at the consensus protocol 
itself. Usually, the reason for such changes is the reinforcement of the 
protocol against a broader scope of adversary attacks, or the optimization of 
some aspect of the system like the transaction throughput, or the storage cost 
etc.
A software update's lifecycle comprises of three important decision points: a) 
What update proposal should be implemented, b) is a specific implementation 
appropriate to be deployed and c) when and how the changes should be activated on the 
blockchain. A fully decentralized approach should decentralize all of these 
three decisions. Indeed, there are already proposals on how to update specific 
blockchain protocols in a decentralized way~\cite{dash,decred,tezos}. Moreover, 
Bingsheng et al. \cite{NDSS:ZhaOliBal19}, proposes a complete treasury system in
order to solve the funding problem for software updates. The decentralization 
of such decisions is usually called in short \emph{decentralized governance}.

This paper does not focus on how to achieve decentralized governance for 
software updates. Indeed, we assume that appropriate decentralized governance processes (e.g., voting, 
delegation of voting, upgrade-readiness signaling etc.) are in place and the 
community has already reached a 
consensus on what specific update should be activated. Moreover, we assume that 
a sufficient percent of honest parties have expressed (e.g. through a signaling 
mechanism) their readiness to upgrade to the new ledger. This is exactly the 
point from where our focus begins. In particular, we deal with the \emph{secure 
activation} of software update changes on the blockchain in a fully 
decentralized setting and essentially 
provide a way to safely transition from the old ledger to the upgraded ledger 
without the need of a trusted third party. Moreover, we define what is a secure 
activation of changes by introducing the notion of \emph{updatable blockchain}. 
To the best of our knowledge, our approach is the first that treats the problem 
of decentralized activation of updates for blockchains in such a formal way 
providing a security definition for updatable blockchain and generic 
constructions (more details will be provided in the next section).


\subsection{Our Contributions}
In our work, we try to define what is a ledger\footnote{With slight abuse of terminology we use the words 
ledger and blockchain interchangeably.} that supports updates and refer to it as an \emph{updatable ledger}.
Then we propose a generic compiler that takes a ledger $\ledger_1$ and turns it into a ledger $\ledgerup$
that can be updated to the code of a ledger $\ledger_2$ under some conditions.
We then propose another compiler that, always starting from a ledger $\ledger_1$ turns it into an updatable ledger that tolerates updates only with respect to ledgers that follow the same consensus rule as $\ledger_1$ but have different blocks structure. All our constructions do not rely on any trusted third party (TTP).
%The latter compiler has been folklore for a long time, but we are the first work that formally argue about its security.

\subsection{Our Techniques} Our definition of updatable ledgers is quite intuitive. We require
an updatable ledger $\ledgerup$ to be secure under the standard definition of security (i.e., it has to enjoy consistency and liveness) but on top of this, it has to support the property of \emph{updatability}. 
This property guarantees that, in the case there are  enough parties that are willing to upgrade 
the code of $\ledgerup$ to the code of a new ledger $\ledger_2$, the honest parties can securely run $\ledger_2$
and preserve the state of $\ledgerup$.

Clearly, (almost) any ledger $\ledger_1$ can be turned into an updatable ledger $\ledgerup$ 
if we can rely on a TTP.
Indeed, in this case the  TTP can issue a genesis block for $\ledger_2$ which incorporates the state of $\ledger_1$ (or just the hash of it), and then the parties that where running $\ledger_1$ can abandon it and start running 
$\ledger_2$ using the genesis block issued by the TTP.

We show how to construct an updatable ledger without relying on a TTP. The starting point for our construction is
a standard ledger $\ledger_1$ that we enhance with the following mechanism. At time $T_0$ (when enough parties are assumed to be willing to update to $\ledger_2$)
a block of $\ledger_1$ is chosen and \emph{translated} into a genesis block for $\ledger_2$. All the parties that wanted to update can now simply run $\ledger_2$ on the chosen genesis block.
This approach clearly requires that there is an efficient way to translate a block of $\ledger_1$ into a block for $\ledger_2$, and this might limit the class of ledgers to which $\ledgerup$ can be updated

Even though the above approach seems to work, there are unfortunately many subtleties that we need to deal with.
The first is that the adversary might be able to see the genesis block for $\ledger_2$ before any other honest parties do, and therefore he can take advantage in the generation of the blocks of $\ledger_2$ thus compromising the security of the system. 
The second issue is that the adversary might influence the choice of the genesis block. Indeed, we do not know how the consensus algorithm of $\ledger_1$ works and what is the power of the adversary in biasing the content of $\ledger_1$'s blocks. We note that this scenario (where there are many candidates blocks and the adversary can decide which block is added to the final chain) is well studied (see~\cite{EC:GarKiaLeo15}) and many blockchain protocols allow this kind of adversarial behaviour (i.e., an adversary can create forks and influence the decision on what fork will become part of the stable chain). 
To tackle these issues, we further shrink the class of ledgers to which $\ledgerup$ can be updated, and require
$\ledger_2$ to retain its security even in the case the genesis block can be seen by the adversary before that the hones parties can see it, and even if the adversary can pick the genesis block from a set of candidate genesis blocks.
Despite being quite general, this compiler has the drawback that the honest parties need to be online during the update. Indeed, if an honest party is offline before $T_0$ and comes online after the update then no security can be guaranteed for this party. Moreover, the throughput of the blockchain is reduced during the updates.  More details are provided in Sec.~\ref{se:first}.

The second scheme that we propose requires $\ledgerup$ and $\ledger_2$ to be the same (i.e.,  they use the same consensus rules) but might have a different block structure. In this case, the update process is even simpler,
the parties, starting from a pre-agreed block index $j$, start extending the state of  $\ledgerup$ using the rules of
$\ledger_2$ even if the block in position $j$ is not stable. That is, it might happen that different honest parties 
start running $\ledger_2$ using a different starting block given that the block $j$ does not belong to the common prefix. We prove that this does not cause issues even in the case when not all the honest parties participate 
to the update (i.e., some honest parties are offline or decided to not participate to the update). 
The advantage of this approach over the first that we have proposed is that we do not require all the honest parties to be online during the update, and the throughput is not affected by the update process. 


